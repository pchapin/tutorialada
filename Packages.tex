\section{Packages}
\label{sec:packages}

A package is a named container into which you can place subprograms, type definitions, and other
entities that are related to each other. Packages can even contain other packages. In Ada
packages are the primary way to organize and manage the various parts of a program.

Packages have two parts: a \newterm{specification} that declares the entities that are visible
to the rest of the program, and a \newterm{body} that contains the implementation of those
entities. The body can also contain private entities that can only be used inside the package.
In most cases the specification is stored in a separate file from the body and can even be
compiled separately. This makes the separation between a package's interface and implementation
more rigorous in Ada than in many languages, encouraging the programmer to separate the process
of software design from construction.

This formal separation makes it easier for different groups of people to work on the design and
construction of a program since those groups would be working with different sets of source
files. At least that is the ideal situation. Alas, package specifications can have a private
section, discussed later, that is technically part of the package's implementation. Thus, in
some cases at least, interface and implementation end up in the same file after all.

As an example, consider a simple package for displaying text on a character mode terminal. The
following example shows what the specification might look like. When using the GNAT compiler,
package specifications are stored in files with a \filename{.ads} extension using the same name
as the name of the package. In this case, the file would be \filename{screen.ads}.

\begin{lstlisting}
package Screen is

   type Row_Type is range 1..25;
   type Column_Type is range 1..80;
   type Color_Type is (Black, Red, Green, Blue, White);

   procedure Set_Cursor_Position
      (Row : in Row_Type; Column : in Column_Type);
   procedure Get_Cursor_Position
      (Row : out Row_Type; Column : out Column_Type);


   procedure Set_Text_Color(Color : in Color_Type);
   procedure Get_Text_Color(Color : out Color_Type);

   procedure Print(Text : String);

end Screen;
\end{lstlisting}

There is normally no executable code in the specification; it only contains
declarations.\footnote{The initializer of a variable, however, can contain executable code.} The
example above shows a few types being declared and then some procedures that make use of those
types. Packages can also contain functions, of course.

Once the specification has been written code that makes use of the package can be compiled. Such
a compilation unit contains a statement such as |with Screen| in its context clause. It can
then, for example, refer to the type |Screen.Row_Type| and call procedure |Screen.Print| as you
might expect. Note that a package specification can contain a context clause of its own if it
needs, for example, types defined in some other package.

The package body, stored in a \filename{.adb} file, contains the implementation of the various
subprograms declared in the specification. The Ada compiler checks for consistency between the
specification and the body. You must implement all subprograms declared in the specification and
you must even use exactly the same names for the parameters to those subprograms. However,
the body can define types and subprograms that are not declared in the specification. Such
entities can only be used inside the package body; they are not visible to the users of the
package. The following example shows an abbreviated version of \filename{screen.adb}:

\begin{lstlisting}
package body Screen is

   -- Not declared in the spec. This is for internal use only.
   procedure Helper is
   begin
     -- Implementation not shown.
   end Helper;

   -- All subprograms declared in spec must be implemented.
   procedure Set_Cursor_Position
      (Row : in Row_Type; Column : in Column_Type) is
   begin
     -- Implementation not shown.
   end Set_Cursor_Position;

   -- etc...

end Screen;
\end{lstlisting}

The package body does not need to include a with clause for its own specification, however it
can |with| other packages as necessary in order to gain access to the resources provided by
those packages.

When building a large program the package specifications can all be written first, before any
real coding starts. The specifications are thus the final output of the design phase of the
software development life cycle. These specifications can be compiled separately to verify that
they are free from syntax errors and that they are internally consistent. The compiler can
verify that all the necessary dependencies are explicitly declared via appropriate |with|
statements, and that all types are used in a manner that is consistent with their declarations.

Once the specifications are in place, the bodies can be written. Because a package body only
depends on the specifications of the packages it uses, the bodies can all be written in
parallel, by different people, without fear of inconsistency creeping into the program.
Programmers could be forbidden to modify any specification files, for example by using
appropriate access control features on a file server or version control system, requiring any
such modifications to first be cleared by a designer. Surprisingly few programming languages
enjoy this level of rigor in the management of code, but Ada was designed for large software
projects and so its features are strong in this area.

\subsection{Package Initialization}

Packages sometimes contain internal variables that need to be initialized in some complex way
before the package can be used. The package author could provide a procedure that does this
initialization, but then there is a risk that it won't get called when it should. Instead, a
package can define some initialization code of arbitrary complexity by introducing an executable
section in the package body. The following example shows how this can be done.

\begin{lstlisting}
with Other;
pragma Elaborate_All(Other);

package body Example is

   Counter : Integer;  -- Internal variable.

   procedure Some_Operation is
   begin
      -- Implementation not shown.
   end Some_Operation;

begin
  Counter := Other.Lookup_Initial("database.vtc.vsc.edu");

end Example;
\end{lstlisting}

% TODO: Is this the first time the word "elborate" is used? If so, it should be introduced.

In this example, an internal variable |Counter| is initialized by calling a function in a
supporting package |Other|. This initialization is done when the package body is elaborated.
Notice that it is necessary in this example for the body of package |Other| to be elaborated
first so that its initialization statements (if any) execute before the elaboration of the body
of package |Example|. If that were not the case, then function |Lookup_Initial| might not work
properly. An Ada program will raise the exception |Program_Error| at run time if packages get
elaborated in an inappropriate order.

To address this potential problem a pragma is included in the context clause of package
|Example|'s body. Pragmas are commands to the compiler that control the way the program is
built. The Ada standard defines a large number of pragmas and implementations are allowed to
define others. In this case the use of pragma |Elaborate_All| forces the bodies of package
|Other|, and any packages that package |Other| uses, to be elaborated before the body of package
|Example|. Note that the |with| clauses will automatically cause the specifications of the
withed packages to be elaborated first, but not necessarily the bodies. In this case it is
essential to control the order in which the package bodies are elaborated. Hence the pragma is
necessary.

Notice that Ada does not provide any facilities for automatically cleaning up a package when the
program terminates. If a package has shutdown requirements a procedure must be defined for this
purpose, and arrangements must be made to call that procedure at an appropriate time. In
embedded systems, one of the major target application areas for Ada, programs often never end.
Thus the asymmetric handling of package initialization and cleanup is reasonable in that
context. Other programming languages do provide ``module destructors'' of some kind to deal with
this matter more uniformly.

\subsection{Child Packages}

Being able to put code into various packages is useful, but in a large program the number of
packages might also be large. To cope with this Ada, like many languages, allows its packages to
be organized into a hierarchy of packages, child packages, grand child packages, and so forth.
This allows an entire library to be contained in a single package and yet still allows the
various components of the library to be organized into different packages (or package
hierarchies) as well. This is an essential facility for any language that targets the
construction of large programs.

As an example, consider a hypothetical data compression library. At the top level we might
declare a package |Compress| to contain the entire library. Package |Compress| itself might be
empty with a specification that contains no declarations at all (and no body). Alternatively one
might include a few library-wide type definitions or helper subprograms in the top level
package.

One might then define some child packages of |Compress|. Suppose that |Compress.Algo| contains
the compression algorithms and |Compress.Utility| contains utility subprograms that are used by
the rest of the library but that are not directly related to data compression. Further child
packages of |Compress.Algo| might be defined for each compression algorithm supported by the
library. For example, |Compress.Algo.LZW| might provide types and subprograms related to the LZW
compression algorithm and |Compress.Algo.Huffman| might provide types and subprograms related to
the Huffman encoding compression algorithm. The following example shows a procedure that wishes
to use some of the facilities of this hypothetical library.

\begin{lstlisting}
with Compress.Algo.LZW;

procedure Hello is
  Compressor : Compress.Algo.LZW.LZW_Engine;
begin
  Compress.Algo.LZW.Process(Compressor, "Compress Me!");
end Hello;
\end{lstlisting}

In this example I assume the LZW package provides a type |LZW_Engine| that contains all the
information needed to keep track of the compression algorithm as it works (presumably a record
of some kind). I also assume that the package provides a procedure |Process| that updates the
compressed data using the given string as input.

Clearly a package hierarchy with many levels of child packages will produce very long names for
the entities contained in those deeply nested packages. This can be awkward, but Ada provides
two ways to deal with that. You have already met one way: the |use| statement. By including use
|Compress.Algo.LZW| in the context clause, the contents of that package are made directly
visible and the long prefixes can be deleted. However, Ada also allows you to rename a package
to something shorter and more convenient. The following example shows how it could look.

\begin{lstlisting}
with Compress.Algo.LZW;

procedure Hello is
  package Squeeze renames Compress.Algo;
  Compressor : Squeeze.LZW.LZW_Engine;
begin
  Squeeze.LZW.Process(Compressor “Compress Me!”);
end Hello;
\end{lstlisting}

The package |Squeeze| is not a real package, but rather just a shorter, more convenient name for
an existing package. This allows you to reduce the length of long prefixes without eliminating
them entirely. Although the Ada community encourages the use of long, descriptive names in Ada
programs, names that are local to a single procedure can sensibly be fairly short since they
have limited scope. Using the renaming facility of the language you can introduce abbreviated
names for packages (or other kinds of entities) with limited scope without compromising the
overall readability of the program.

Notice that the optimal names to use for entities in a package will depend on how the package
itself is used. In the example above the package |LZW| provides a type |LZW_Engine|. In cases
(such as shown) where fully qualified names are used the ``LZW'' in the name |LZW_Engine| is
redundant and distracting. It might make sense to just name the type |Engine| yielding a fully
qualified name of |Compress.Algo.LZW.Engine|. On the other hand, if use statements are used the
name |Engine| by itself is rather ambiguous (what kind of engine is that?). In that case it
makes more sense to keep the name |LZW_Engine|. Different programmers have different ideas about
how much name qualification is appropriate and it leads to differences in the way names are
selected.

The problem is that the programmer who writes a package is often different from the programmer
who uses it and so incompatible naming styles can arise. Ada's renaming facility gives the using
programmer a certain degree of control over the names that actually appear in his or her code,
despite the names selected by the author of a library. The following example shows a more
radical approach to renaming.

\begin{lstlisting}
with Compress.Algo.LZW;

procedure Hello is
  subtype LZW_Type is Compress.Algo.LZW.LZW_Engine;
  procedure Comp(Engine : LZW_Type; Data : String)
       renames Compress.Algo.LZW.Process;
  Compressor : LZW_Type;
begin
  Comp(Compressor "Compress Me!");
end Hello;
\end{lstlisting}

Notice that while packages and subprograms (and also objects) can be renamed, types can not be
renamed in this way. Instead you can introduce a subtype without any constraints as a way of
effectively renaming a type.

In this simple example, the renaming declarations bulk up the code more than they save. However,
in a longer and more complex situation they can be useful. On the other hand, you should use
renaming declarations cautiously. People reading your code may be confused by the new names if
it is not clear what they represent.

