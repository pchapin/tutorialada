%%%%%%%%%%%%%%%%%%%%%%%%%%%%%%%%%%%%%%%%%%%%%%%%%%%%%%%%%%%%%%%%%%%%%%%%%%%
% FILE    : Huffman.tex
% SUBJECT : A Huffman encoding case study in Ada.
% AUTHOR  : (C) Copyright 2015 by Peter Chapin <PChapin@vtc.vsc.edu>
%
%%%%%%%%%%%%%%%%%%%%%%%%%%%%%%%%%%%%%%%%%%%%%%%%%%%%%%%%%%%%%%%%%%%%%%%%%%%

% ++++++++++++++++++++++++++++++++
% Preamble and global declarations
% ++++++++++++++++++++++++++++++++
\documentclass{scrreprt}

% --------
% Packages
% --------
\usepackage{amsthm}
\usepackage{fancyvrb}
\usepackage[pdftex]{graphicx}
\usepackage{listings}
\usepackage{hyperref}
\usepackage{url}

\newcommand{\SPARK}{\textsc{Spark}}   % For ease of small capping SPARK.

% Redefine the underscore command so latex will break words at underscores. Found this trick at
% http://mrtextminer.wordpress.com/2009/02/26/break-a-long-word-containing-underscores-in-latex/
%
\let\underscore\_
\newcommand{\myunderscore}{\renewcommand{\_}{\underscore\hspace{0pt}}}

% Issue the changed underscore command to the whole document.
\myunderscore

% Define a new 'leo' style for the package that will use a smaller font for URLs. It also allows
% LaTEX to break a URL across lines. From Jérôme Hugues.
%
\makeatletter
\def\url@leostyle{%
  \@ifundefined{selectfont}{\def\UrlFont{\sf}}{\def\UrlFont{\small\ttfamily}}}
\makeatother

% Now actually use the newly defined style.
\urlstyle{leo}

%% The following are settings for the listings package. See the listings package documentation
%% for more information.
%%
%\lstset{language=C,
%        basicstyle=\small,
%        stringstyle=\ttfamily,
%        commentstyle=\ttfamily,
%        xleftmargin=0.25in,
%        showstringspaces=false}

% Define listing parameters for Ada 2012. Base it on Ada 2005.
\lstdefinelanguage{Ada2012}[2005]{Ada}
{
  morekeywords={some},
  sensitive=false,
  breaklines=false,
  showstringspaces=false,
  xleftmargin=0.25in,
  basicstyle=\small\sffamily,
  columns=flexible,
}
\lstset{language=Ada2012, showlines=true}

% Wide equivalent of the listings package lstset. The parameter is the amount to indent the code
% (may be negative).
%
\lstnewenvironment{widelisting}[1]
   {\lstset{language=Ada2012,xleftmargin=#1}}
   {}
   
% Wide listing from a file using the Listings package lstinputlisting. First parameter is
% path name to file (can be relative). Second parameter is the amount to indent the code (may
% be negative).
%
\newcommand{\wideinputlisting}[2]{%
   \lstinputlisting[language=Ada2012,xleftmargin=#2]{#1}}

% ------------------
% Layout adjustments
% ------------------
% Avoid changing the layout; it is usually set via the document class (or in packages).
%
% \pagestyle{headings}
%
% The following create block paragraphs (no indentation, extra separation).
% \setlength{\parindent}{0em}
% \setlength{\parskip}{1.75ex plus0.5ex minus0.5ex}


% ------
% Macros
% ------
\newcommand{\command}[1]{\texttt{#1}}  % For commands typed at a console/terminal prompt.
\newcommand{\filename}[1]{\texttt{#1}} % For the names of files in the file system.
\newcommand{\newterm}[1]{\textit{#1}}  % For the first time introduction of a term.

% +++++++++++++++++++
% The document itself
% +++++++++++++++++++
\begin{document}

% ----------------------
% Title page information
% ----------------------
\titlehead{\centering\includegraphics[scale=0.05]{../Ada-Mascot.pdf}}
\title{Huffman Encoding--A Case Study}
\author{Peter Chapin\thanks{pchapin@vtc.edu}\\
  Vermont Technical College}
\date{Generated: \today}
\maketitle

\tableofcontents

%\lstMakeShortInline[basicstyle=\ttfamily]|
\lstMakeShortInline|

\chapter{Huffman Encoding}
\label{chapt:huffman-encoding}

To better understand the features of Ada, it is helpful to work through a small but realistic
program using the language. In this section I will design and implement a simple file
compression utility that uses the Huffman encoding compression algorithm. Although Huffman
encoding is not particularly effective by itself, it is good enough to provide a usable utility
for some applications. It also provides a good balance of complexity and simplicity for this
exercise.

\section{The Algorithm}

Huffman encoding takes advantage of the fact that in most files some byte values are much more
common than others. By assigning short binary codes to the most commonly occurring values the
file can be made smaller. One consequence, however, of assigning short codes to some values is
that other values must necessarily be assigned long binary codes. This is necessary in order to
properly distinguish between all possible values.

For example, assume that the most commonly occurring value is assigned a single bit code of 1.
All other values must necessarily have codes starting with 0 so that they can be properly
recognized. There could be as many as 255 such values, requiring 8 additional bits (for a total
of nine bits) to distinguish those values among themselves.

Huffman encoding finds a way to assign variable length codes to the byte values to minimize the
overall file size. Specifically it uses byte value frequency information to make the code
assignments. That is, commonly occurring bytes are assigned relatively short codes and
infrequently occurring bytes are assigned correspondingly longer codes. Itt can be shown that
Huffman encoding is optimal; no other compression method that does not take advantage of
correlations can provide more compression. Huffman encoding is also an example of a greedy
algorithm and is often discussed in algorithms textbooks as such.

The algorithm has three phases:

\begin{enumerate}

\item Scan the input and count the number of times each byte value occurs.

\item Construct a \newterm{code tree} that reflects the relative frequency of the byte values
  and assign the variable length codes to each value.

\item Rewrite the input substituting the codes assigned in step \#2 for the byte values.

\end{enumerate}

One significant disadvantage to Huffman encoding is that the input must be scanned completely to
do the frequency analysis before any output can be written. In some applications the entire
input is not known initially (for example: streaming data or user input) and that is a major
problem for this method. In such cases the compressor can sometimes make an educated guess about
the relative frequencies of the byte values, based perhaps on past observations of similar
input, but the results will then be only approximate. This is not an issue in this case; I am
interested in compressing files and I have access to the entire file before I have to compress
any of it.

The output of the first phase is, conceptually, a table with one entry for each byte value (256
entries in all). Associated with each entry is a count of the number of times that value
appeared in the input. For example, assume the algorithm is applied to a file containing only
the letters `A' through `E'. The table output by the first phase might look as follows:

\textit{Show table\ldots}

The table elements become the leaves of a tree. Pseudo code for the tree building algorithm is
shown below. Each pass of the while loop combines the counts from the active nodes with the
smallest and next smallest count. Those nodes are then removed from the active list and the new
node is added to the active list (where its combined count is then considered in later passes).
Thus each pass of the while loop reduces the size of the active list by one. Eventually only a
single active node is left and that node is the root of the code tree.

\begin{Verbatim}
WHILE <There is more than one active node> LOOP
  <Find nodes x, y with the smallest and next smallest counts>
  <Create a new node z that combines the count values;
     Point z at x and y>
  <Remove x and y from the set of active nodes>
  <Add z to the set of active nodes>
END
\end{Verbatim}

For example using the initial counts as shown above, the result after the first past of the
while loop is as follows:

\textit{Show figure\ldots}

In the next pass of the while loop the nodes with counts 42 (smallest) and 45 (next smallest)
will be combined. After the loop finishes executing in the sample above the code tree looks
like:

\textit{Show figure\ldots}

Notice that the nodes with large counts, such as the node for `A' are combined late in the
process. Such nodes thus end up being a short distance from the root of the code tree. This is
the basis for their short codes. Notice also that the count in the root node ends up being the
same as the total number of bytes in the input. This can be used as a validity check on the code
tree construction process.

After the code tree has been constructed, codes are assigned to the original byte values by
assigning a 1 or 0 bit to each link in the code tree. The assignment can be arbitrary without
affecting the correctness of the result, but for the sake of choosing something, I will assign a
1 bit to the child node with the larger count value.  After the bit assignments, the code tree
becomes:

\textit{Show figure\ldots}

The code for a byte value is then read from the root, assigning bits from left to right in the
code as each link in the code tree is traversed. In my example this yields the codes as shown
below:

\textit{Show table of assigned codes\ldots}

Notice how the codes for the most commonly occurring byte values (`A' and `C') are short while
the codes for the most infrequently occurring byte values (`B' and `D') are long. Notice also
that there is no ambiguity in the encoding. For example, a bit sequence such as 100110111000 can
only be interpreted as the byte sequence ``ABCAAE.'' Notice also that if ordinary 8 bit bytes
are used the string ``ABCAAE'' would consume 6 bytes of space. However, the bit sequence
100110111000 is only 12 bits (1.5 bytes) long.

\section{Implementation Notes}

Before looking at the detailed design of the Huffman compression program I want to point out a
few issues of interest. Keep these issues in mind when you review the program's design and the
source code of the program itself.

\begin{enumerate}

\item Both the input and output files have to be processed as raw binary bytes without
  structure. To make the program general, it should not concern itself with the meaning of the
  input file's contents.

\item The construction of the code tree can't start until the input file has been fully
  analyzed. Similarly no output can be written until the code tree is fully constructed and code
  sequences have been assigned. This means there is little opportunity for parallelism among
  these phases of execution.

\item The pointers in a code tree node sometimes point at other code tree nodes (presumably
  allocated dynamically) and sometimes point at the original table of counter values.

\item An auxiliary data structure of some kind will be necessary to keep track of which code
  tree nodes are currently ``active.''

\item Since the code sequences have variable length, writing the output file is tricky. A way
  must be provided to write partial bytes and to write long bit sequences that span multiple
  bytes.

\item For the compressed file to be decompressed properly the receiver will need to build
  exactly the same code tree as the sender. One way to allow the receiver to do this would be
  for the sender to transmit the original counts in a header on the compressed file. If 32 bit
  counts are used, this adds 1024 bytes of overhead to the compressed file.

\item In a realistic example some code sequences can be very long (one or two hundred bits in an
  extreme case). It may be appropriate to use a variable length data structure to handle them.

\end{enumerate}

\lstDeleteShortInline|

% At some point these references must make it into the bibliography for this document.
%
% 1. WikiBook on Ada: http://en.wikibooks.org/wiki/Ada_Programming. This is one of the better of
% the WikiBooks. As always anything on-line is a work in progress. Nevertheless many aspects of
% Ada are well described.
%
% 2. Ada Newsgroup: news://comp.lang.ada. An active newsgroup where you can post questions and
% talk with Ada experts.
%
% 3. Ada IRC channel: irc://irc.freenode.net/#ada. Although often quiet there are usually a few
% people there who can answer questions.
%
% 4. There are several Ada web sites. The Ada Information Clearinghouse (http://www.adaic.com/),
% site is a good place to start.
%
%\bibliography{references}
%\bibliographystyle{plain}

\end{document}

