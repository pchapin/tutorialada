\section{Access Types}

Ada, like many languages, allows you to create objects that refer to other objects. It also
allows you to create objects that have a lifetime extending beyond that of the scope in which
they are created. Both of these capabilities are important and many programming techniques
depend on them. C and C++ allow the programmer to use arbitrary pointers, implemented as simple
memory addresses, in any way desired. While this is very powerful, it is also very dangerous. C
and C++ programs suffer from many bugs and security vulnerabilities related to the unrestricted
use of pointers in those languages.

In Ada pointer types are called \newterm{access types}. Like C and C++ pointer variables,
\newterm{access variables} can be manipulated separately from the objects to which they point.
However, unlike C and C++ pointer types, access types in Ada have a number of restrictions on
their use that are designed to make them safer.

Actually, the history of access types in Ada is quite interesting. In Ada 83 access types were
very limited but also very safe. However, experience with Ada showed that the limitations were
too great. Ada 95 removed some of the limitations while still managing to keep the safety (at
the expense of complicating the language). Yet even after these changes, access types were still
not as flexible as desired. Ada 2005 removed yet more limitations but required, in certain
cases, run time checking to be done to ensure safety. In this tutorial I will not describe these
issues in detail. My focus here is on the basic use of access types.

Access types can be named or anonymous. I will only consider named access types; anonymous
access types have special properties that are outside the scope of this tutorial.

For example the declaration:

\begin{lstlisting}
type Integer_Access is access Integer;
\end{lstlisting}

\noindent declares |Integer_Access| as a type suitable for accessing, or pointing at, integers.
Notice that in this tutorial I have used a suffix of ``Type'' when naming a type. In this case,
however, I use a suffix of ``Access'' to emphasize the nature of the access type. This is, of
course, just a convention.

Once the access type has been declared, variables of that type can then be declared in the usual
way.

\begin{lstlisting}
P : Integer_Access;
\end{lstlisting}

Ada automatically initializes access variables to the special value |null| if no other
initialization is given. Thus access variables are either null or they point at some real
object. Indeed, the rules of Ada are such that, under appropriate circumstances, dangling
pointers are not possible. Access variables can be copied and compared like any other variable.
For example if |P1| and |P2| are access variables than |P1 = P2| is true if they both point at
the same object. To refer to the object pointed at by an access variable you must use the
special |.all| operation.

\begin{lstlisting}
P.all := 1;
\end{lstlisting}

The |.all| syntax plays the same role in Ada as the indirection operator (the *) plays in C and
C++. The use of |.all| may seem a little odd in this context, but understand that most of the
time access types are pointers to aggregate entities such as arrays or records. In that case,
the components of the array or record pointed at can be accessed directly by using the component
selection operation on the access variable itself. This is shown as follows:

\begin{lstlisting}
type Date is
  record
    Day, Month, Year : Integer;
  end record;
type Date_Access is access Date;

D1, D2 : Date_Access;

...

D1.Day := 1;      -- Accesses the Day member of referenced Date.
D1     := D2;     -- Causes D1 to point at same object as D2.
D1.all := D2.all; -- Copies Date objects.
\end{lstlisting}

In this case the |.all| syntax is very natural. It shows that you are accessing all the members
of the referenced object at the same time. It is important to notice that Ada normally
``forwards'' operations applied to the access variable to the referenced object. Thus |D1.Day|
in the example means the |Day| component of the object pointed at by |D1|. Only operations that
are meaningful for the access type itself are not forwarded. Thus |D1 := D2| copies the access
values. To copy the objects pointed at by the access variables one must use the |.all| syntax. I
should also note that syntax such as |D1.all.Day|, while verbose, is also legal. The |.all|
dereferences |D1|. The result is a date record so the selection of the |Day| component is
meaningful.

Access variables that refer to arrays also allow the normal array operations to be forwarded to
the referenced object as the following example illustrates:

\begin{lstlisting}
type Buffer_Type is array(0..1023) of Character;
type Buffer_Access is access Buffer_Type;
B1, B2 : Buffer_Access;

...

B1     := B2;     -- Copies only the access values.
B1.all := B2.all  -- Copies arrays.
B1(0)  := 'X'     -- Forwards index operation.
for I in B1'Range loop    -- Forwards 'Range attribute.
  ...
end loop;
\end{lstlisting}

Because of forwarding, using access types is generally quite convenient in Ada. However, you
must keep in mind that operations that are meaningful for the access types themselves will be
applied directly to the access variable and are not forwarded.

So far we've seen how to declare and use access types. How does one get an access variable to
point at another object in the first place? In Ada 83 there was only one way: using the new
operation. For example:

\begin{lstlisting}
P := new Integer'(0);
\end{lstlisting}

dynamically allocates an integer, initializes that integer to zero, and then assigns the
resulting access value to |P|. Here I assume |P| has an access to integer type. Notice that the
argument to new has the form of a qualified expression (the apostrophe is required). Also as
with dynamically allocated objects in other languages, the lifetime of the object created in
this way extends beyond the lifetime of the access variable used to point at it.

\subsection{Garbage Collection?}

In many languages dynamically allocated objects are automatically reclaimed when they can no
longer be accessed. For example, in Ada parlance, when all access variables pointing at an
object go out of scope the object that was pointed at by those variables can no longer be
referenced and the memory it uses should be made available again. The process of reclaiming such
memory is called \newterm{garbage collection}.

In Ada garbage collection is optional. Implementations are allowed to provide it, but they are
not required to do so. This may seem surprisingly wishy-washy for a language that endeavors to
support reliable and portable programming. The problem is that Ada also endeavors to support low
level embedded systems and real time programming. In such environments garbage collection is
widely considered problematic. It is difficult to meet real time objectives if a complex garbage
collection algorithm might run at any time. Also the space overhead of having a garbage
collector in the run time system might be unacceptable for space constrained embedded devices.
Advances in garbage collection technology and machine capabilities have made some of these
concerns less pressing today than they were when Ada was first designed. However, these matters
are still important. Thus Ada allows, but does not require garbage collection.

This presents an immediate problem for programmers interested in portability. If the
implementation does not collect its garbage and the programmer takes no steps to manually
reclaim allocated objects, the program will leak memory. This is a disaster for long running
programs like servers. Thus for maximum portability one must assume that garbage collection is
not done and take steps accordingly. In fact, most Ada implementations do not provide garbage
collection, so this is a very realistic concern.

The Ada library provides a generic procedure named |Unchecked_Deallocation| that can be used to
manually deallocate a dynamically allocated object. Unfortunately the use of
|Unchecked_Deallocation| can violate important safety properties the language otherwise
provides. In particular, if you deallocate an object while an access variable still points at
it, any further use of that access variable will result in erroneous behavior. As a service
|Unchecked_Deallocation| will set the access variable you give it to |null| causing future use
of that access variable to result in a well defined exception. However, there might be other
access variables that point at the same object and |Unchecked_Deallocation| cannot, in general,
know about all of them.

If your program never uses |Unchecked_Deallocation| then all access variables are either |null|
or point at a real object; dangling pointers are impossible. However, your program might also
leak memory if the Ada implementation does not provide garbage collection. Thus in most real
programs |Unchecked_Deallocation| is used.

This is an example of where Ada compromises safety for the sake of practical reality. In fact,
there are several other ``Unchecked'' operations in Ada that are used to address certain
practical concerns and yet introduce the possibility of unsafe programs. Since every such
operation starts with the word ``Unchecked'' it is an simple matter to search an Ada program for
occurrences of them. This feature makes reviewing the unchecked operations easier.

\textit{Show an example of Unchecked\_Deallocation\ldots}

