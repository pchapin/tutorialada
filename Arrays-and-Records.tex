\section{Arrays and Records}

Unlike in C, arrays in Ada are first class objects. This means you can assign one array to
another, pass arrays into subprograms and return arrays from functions. Every array has a type.
However, the type of an array does not need to be named. For example:

\begin{lstlisting}
Workspace : array(1 .. 1024) of Character;
\end{lstlisting}

\noindent defines |Workspace| to be an array of characters indexed using the integers 1 through
1024. Note that although |Workspace| has an array type, the name of that type is not specified.

In fact, you can define arrays using any discrete subtype for the index. Recall that |1 .. 1024|
is really an abbreviation for an integer subtype specification. It is not possible to access an
array out of bounds for exactly the same reason it's not possible to store an out-of-bounds
value into a variable. The compiler raises the |Constraint_Error| exception in precisely the
same way.

The following example:

\begin{lstlisting}
procedure Example is
  type Day_Type is (Sun, Mon, Tue, Wed, Thu, Fri, Sat);
  Work_Hours : array(Day_Type) of Natural;

  function Adjust_Overtime
    (Day : Day_Type; Hours : Natural) return Natural is
  begin
    -- Not shown.
  end Adjust_Overtime;

begin
  Work_Hours := (0, 8, 8, 8, 8, 0);
  for Day in Day_Type loop
    Work_Hours(Day) := Adjust_Overtime(Day, Work_Hours(Day));
  end loop;
end Example;
\end{lstlisting}

This example illustrates several important features.

\begin{enumerate}

\item Arrays are accessed using parenthesis and not square brackets as in C family languages.
  Thus array access has a syntax similar to that of a function call. An array of constant values
  and can be replaced with a function later without clients needing to be edited.

\item Any discrete subtype, including enumeration subtypes, can be used for an array index. In
  the preceding example the array |Work_Hours| has elements |Work_Hours(Sun)| and so forth.

\item The nested function |Adjust_Overtime| uses a type defined in the enclosing procedure. This
  is perfectly acceptable. The visibility of all names is handled uniformly.

\item It is possible to use an \newterm{array aggregate} in a expression where an array is
  expected. |Work_Hours| is assigned a value that specifies every element of the array in one
  step. The compiler deduces the type of the aggregate from its context.

\item Notice that in the preceding example the same subtype is used for both the loop index
  variable and the array index. This is a common situation, and it means the compiler can
  optimize out the run time checks normally required when accessing an array element. Since the
  value of |Day| can't possibly be outside the range of allowed array indexes, there is no need
  to check that |Work_Hours(Day)| is in bounds.

\end{enumerate}

In many cases it is appropriate to give a name to an array type. This is necessary if you want
to assign one array to another, compare two arrays for equality, or pass arrays into or out of
subprograms. These operations require matching types, and with no way to talk about the type of
array it isn't possible to declare two arrays with the same type. The following example
illustrates:

\begin{lstlisting}
procedure Example is
  type Buffer_Type is array(0 .. 1023) of Character;
  B1 : Buffer_Type;
  B2 : Buffer_Type;
  B3 : array(0 .. 1023) of Character;
begin
  B1 := B2;  -- Fine. Entire array assigned.
  B1 := B3;  -- Error. Types don't match. B3 has anonymous type.
end Example;
\end{lstlisting}

Because subtypes are defined dynamically the size and bounds of an array can also be defined
dynamically. For example a declaration such as |A : array(1 .. N) of Natural| is allowed even if
the value of |N| is not known at compile time. For example, it could be a subprogram parameter.
This feature means that many places where dynamic allocation is necessary in, for example, C++
can be handled in Ada without the use of explicit memory allocators.\footnote{In fairness, C++
  programmers routinely use library containers that hide the dynamic memory allocation they
  require.}

\subsection{Unconstrained Array Types}

One problem with the arrays I've described so far is that it is not possible to write a
procedure that takes an array of unknown size. Arrays of different sizes have different types
and strong typing will prevent them from being mixed. To get around this problem Ada has the
concept of unconstrained array types. Such a type does not specify the bounds on the array as
part of the type, allowing arrays with different bounds to be in the same type. However, the
bounds must be specified when an actual array object is declared, so the compiler knows how much
memory to allocate for the object. The following example illustrates:

\begin{lstlisting}
procedure Unconstrained_Array_Example is
  type Buffer_Type is array(Integer range <>) of Character;
  B1 : Buffer_Type;  -- Error. Must specify array bounds.
  B2 : Buffer_Type( 0 .. 15);  -- Okay.
  B3 : Buffer_Type( 0 .. 15);
  B4 : Buffer_Type(16 .. 31);  -- Fine.
  B5 : Buffer_Type( 0 .. 63);  -- No problem.

  procedure Process_Buffer(Buffer : in Buffer_Type) is
  begin
    for I in Buffer'Range loop
      -- Do something with Buffer(I)
    end loop;
  end Process_Buffer;

begin
  B2 := B3;  -- Fine. Types match and bounds compatible.
  B2 := B4;  -- Fine! Types match and lengths identical.
  B2 := B5;  -- Constraint_Error. Lengths don't match.

  Process_Buffer(B2);  -- Fine.
  Process_Buffer(B4);  -- Fine.
  Process_Buffer(B5);  -- Fine.
end Unconstrained_Array_Example;
\end{lstlisting}

Again there are several points to make about this example.

\begin{enumerate}

\item The symbol |<>| (pronounced ``box'') is intended to be a placeholder for information that
  will be filled in later. In this case it specifies that the index bounds on the array type
  |Buffer_Type| is unconstrained.

\item It is illegal to declare a variable with an unconstrained type without providing
  constraints. This is why the declaration of |B1| is an error. However, the missing array
  bounds are specified in the other declarations. Notice that it is not necessary for array
  bounds to start at zero or one. Any particular value is fine as long as it is part of the
  index type mentioned in the array declaration.

\item It is fine to declare a subprogram, such as |Process_Buffer|, taking an unconstrained
  array type as a parameter. However, such a subprogram can't safely access the elements of the
  given array using specific index values like 1 or 2 because those values might not be legal
  for the array. Instead, you need to use special array attributes. For example, |Buffer'First|
  is the first index value that is valid for array |Buffer|, and similarly for
  |Buffer'Last|\footnote{These attributes only return valid indexes if the array has a non-zero
    length}. The attribute |Range| is shorthand for |Buffer'First .. Buffer'Last| and is quite
  useful in for loops as the example illustrates.

\item You might find it surprising that the assignment |B2 := B4| is legal since the array
  bounds do not match. However, the two arrays have the same length, so corresponding elements of
  the arrays are assigned. This is called \newterm{sliding semantics} because you can imagine
  sliding one array over until the bounds do match.

\item All the calls to |Process_Buffer| are fine. Inside the procedure |Buffer'Range| adapts
  itself to whatever array it is given and, provided the code is written with this in mind, the
  procedure works for arrays of any size.

\end{enumerate}

\subsection{Records}

In C family languages composite objects containing components with different types are called
structures. In Ada, and in many other languages, they are called records. All record types must
have a name and thus must be declared before any objects of that type can be created. An example
follows:

\begin{lstlisting}
procedure Example is
  type Date is
    record
      Day   : Integer range 1 .. 31;
      Month : Integer range 1 .. 12;
      Year  : Natural;
    end record;

  Today    : Date := (Day => 1, Month => 7, Year => 2007);
  Tomorrow : Date;
begin
  Tomorrow := Today;
  Tomorrow.Day := Tomorrow.Day + 1;
end Example;
\end{lstlisting}

This procedure defines a type |Date| as a collection of three named components. Notice that the
|Day| and |Month| components are defined as subtypes of |Integer| without the bother of naming
the subtypes involved (anonymous subtypes are used). |Today| is declared to be a |Date| and
initialized with a \newterm{record aggregate}. In this case named association is used to make it
clear which component gets which initial value, however positional association is also legal.

Records can be assigned as entire units and their components accessed using the dot operator.
Notice that if |Tomorrow.Day + 1| creates a value that is outside the range |1 .. 31|, a
|Constraint_Error| exception will be raised, as usual when that out of bounds value is assigned
to |Tomorrow.Day|. Obviously a more sophisticated procedure would check for this and adjust the
month as necessary (as well as deal with the fact that not all months are 31 days long).

% TODO: Add some exercises.

