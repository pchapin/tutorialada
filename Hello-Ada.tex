\section{Hello, Ada}
\label{sec:hello-ada}

Whenever you begin to study a new language or software development environment you should start
by writing the most trivial program possible in that language or environment. Thus I will begin
this tutorial with a short but complete program that displays the string ``Hello, Ada!'' on the
console.

\label{lst:hello-ada}
\begin{lstlisting}
with Ada.Text_IO;

procedure Hello is
begin
   Ada.Text_IO.Put_Line("Hello, Ada!");
end Hello;
\end{lstlisting}

\noindent The program starts with a \newterm{context clause} consisting of a |with| statement.
The context clause specifies the packages that will be used in this particular compilation unit.
Although the compilation unit above contains just a single procedure, most Ada code is in
packages. The standard library components are in child packages of the package |Ada|. In this
case we will be using facilities in the child package |Ada.Text_IO|.

The main program is the procedure named |Hello|. The precise name used for the main program can
be anything; exactly which procedure is used as the program's entry point is specified when the
program is compiled. The procedure consists of two parts: the part between |is| and |begin| is
called the declarative part. Although empty in this simple case, it is here where you would
declare local variables and other similar things. The part between |begin| and |end| constitute
the executable statements of the procedure. In this case, the only executable statement is a
call to procedure |Put_Line| in package |Ada.Text_IO|. As you can probably guess from its name,
|Put_Line| prints the given string onto the program's standard output device. It also terminates
the output with an appropriate end-of-line marker.

Notice how the name of the procedure is repeated at the end. This is optional, but considered
good practice. In more complex examples the readability is improved by making it clear exactly
what is ending. Notice also the semicolon at the end of the procedure definition. C family
languages do not require a semicolon here so you might accidentally leave it out.

Spelling out the full name |Ada.Text_IO.Put_Line| is rather tedious. If you wish to avoid it you
can include a |use| statement for the |Ada.Text_IO| package in the context clause (or in the
declarative part of |Hello|). Where the |with| statement makes the names in the withed package
\newterm{visible}, the |use| statement makes them \newterm{directly visible}. Such names can
then be used without qualification as shown below:

\begin{lstlisting}
with Ada.Text_IO; use Ada.Text_IO;

procedure Hello is
begin
   Put_Line("Hello, Ada!");
end Hello;
\end{lstlisting}

\noindent Many Ada programmers do this to avoid having to write long package names all the time.
However, indiscriminate use of |use| can make it difficult to understand your program because it
can be hard to tell in which package a particular name has been declared.

Ada is a case insensitive language. Thus identifiers such as |Hello|, |hello|, and |hElLo| all
refer to the same entity. It is traditional in modern Ada source code to use all lower case
letters for reserved words. For all other identifiers, capitalize the first letter of each word
and separate multiple words with an underscore.\footnote{This style is often called ``title
  case.''} The Ada language does not enforce this convention but it is a well established
standard in the Ada community so you should follow it.

Before continuing I should describe how to compile the simple program above. I will assume you
are using the freely available GNAT compiler. This is important because GNAT requires specific
file naming conventions that you must follow. These conventions are not part of the Ada language
and are not necessarily used by other Ada compilers. However, GNAT depends on these conventions
in order to locate the files containing the various compilation units of your program.

The procedure |Hello| should be stored in a file named \filename{hello.adb}. Notice that the
name of the file must be in lower case letters and must agree with the name of the procedure
stored in that file. The \filename{adb} extension stands for ``Ada body.'' This is in contrast
with Ada specification files that are given an extension of \filename{ads}. You will see Ada
specifications when I talk about packages in Section~\ref{sec:packages}. I will describe other
GNAT file naming requirements at that time.

To compile \filename{hello.adb}, open a console (or terminal) window and use the gnatmake
command as follows

\begin{Verbatim}
> gnatmake hello.adb
\end{Verbatim}

\noindent The \command{gnatmake} command will compile the given file and link the resulting
object code into an executable producing, in the above example, \filename{hello.exe} (on
Windows). You can compile Ada programs without \command{gnatmake} by running the compiler and
linker separately. There are sometimes good reasons to do that. However, for the programs you
will write as a beginning Ada programmer, you should get into the habit of using
\command{gnatmake}.

Note that GNAT comes with a graphical Ada programming environment named GPS (GNAT Programming
Studio). GPS is similar to other modern integrated development environments such as Microsoft's
Visual Studio or Eclipse. Feel free to experiment with GPS if you are interested. However, the
use of GPS is outside the scope of this tutorial.

When the compilation has completed successfully you will find that several additional files have
been created. The files \filename{hello.o} and \filename{hello.exe} (on Windows) are the object
file and executable file respectively. The file \filename{hello.ali} is the Ada library
information file. This file is used by GNAT to implement some of the consistency checking
required by the Ada language. It is a plain text file; feel free to look at it. However, you
would normally ignore the \filename{ali} files. If you delete them, they will simply be
regenerated the next time you compile your program.

\subsection*{Exercises}

\begin{enumerate}
\item Enter the trivial ``Hello, Ada'' program on page~\pageref{lst:hello-ada} into your system.
  Compile and run it.

\item Make a minor modification to the trivial program that results in an error. Try compiling
  the program again. Try several different minor errors. This will give you a feeling for the
  kinds of error messages GNAT produces.

\item Experiment with the |use| statement. Try calling |Put_Line| without specifying its
  package, both with and without the |use| statement. Try putting the |use| statement inside the
  declarative part of the procedure. Try putting the |with| statement inside the declarative
  part of the procedure.
\end{enumerate}

